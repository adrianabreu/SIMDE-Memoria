%%%%%%%%%%%%%%%%%%%%%%%%%%%%%%%%%%%%%%%%%%%%%%%%%%%%%%%%%%%%%%%%%%%%%%%%%%%%%%%
% Chapter 4 :  Objetivos y fases
%%%%%%%%%%%%%%%%%%%%%%%%%%%%%%%%%%%%%%%%%%%%%%%%%%%%%%%%%%%%%%%%%%%%%%%%%%%%%%%

%++++++++++++++++++++++++++++++++++++++++++++++++++++++++++++++++++++++++++++++
\section{Objetivos}
\label{4:sec1}

%++++++++++++++++++++++++++++++++++++++++++++++++++++++++++++++++++++++++++++++
\section{Fases}
\label{4:sec2}

El desarrollo del trabajo se ha divido en cuatro fases principales desde un
primer momento.

\begin{enumerate}
   \item \textbf{Migración del “núcleo de la aplicación”}: Es decir las estructuras básicas 
   comunes a las máquinas y el punto de entrada para la generación de las estructuras 
   correspondientes, es decir, el analizador léxico de MIPS.

   \item \textbf{Migración de la máquina superescalar}: El proceso completo de reescribir 
   todas las estructuras de la máquina y su correcto funcionamiento en javascript.

   \item \textbf{Desarrollo de la interfaz}: El desarrollo tanto de un diseño remodelado 
   con las tecnologías web, como de componentes que gestionen correctamente su propio estado.
   
   \item \textbf{Integración interfaz-máquina}: Realizar las conexiones necesarias entre el 
   código desarrollado y la interfaz que permitan permitir le uso del simulador por parte del usuario.

\end{enumerate}

