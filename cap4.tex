%%%%%%%%%%%%%%%%%%%%%%%%%%%%%%%%%%%%%%%%%%%%%%%%%%%%%%%%%%%%%%%%%%%%%%%%%%%%%%%
% Chapter 4 :  Objetivos y fases
%%%%%%%%%%%%%%%%%%%%%%%%%%%%%%%%%%%%%%%%%%%%%%%%%%%%%%%%%%%%%%%%%%%%%%%%%%%%%%%

%++++++++++++++++++++++++++++++++++++++++++++++++++++++++++++++++++++++++++++++
\section{Objetivos}
\label{4:sec1}

Al comienzo de este trabajo fin de grado se tenía una visión abstracta de lo que
se quería conseguir. En las primeras reuniones con el director, se concentraron 
los objetivos del proyecto.

\begin{enumerate}

\item Migrar la lógica de la aplicación a la web.

\item Proveer de una nueva interfaz a la aplicación de SIMDE.

\item Recuperar la documentación de SIMDE.

\end{enumerate}

%++++++++++++++++++++++++++++++++++++++++++++++++++++++++++++++++++++++++++++++
\section{Fases}
\label{4:sec2}

El desarrollo del trabajo se ha divido en cuatro fases principales desde un
primer momento.

\begin{enumerate}
   \item \textbf{Migración del “núcleo de la aplicación”}: Es decir las estructuras básicas 
   comunes a las máquinas y el punto de entrada para la generación de las estructuras 
   correspondientes, es decir, el analizador léxico de MIPS.

   \item \textbf{Migración de la máquina superescalar}: El proceso completo de reescribir 
   todas las estructuras de la máquina y su correcto funcionamiento en javascript.

   \item \textbf{Desarrollo de la interfaz}: El desarrollo tanto de un diseño remodelado 
   con las tecnologías web, como de componentes que gestionen correctamente su propio estado.
   
   \item \textbf{Integración interfaz-máquina}: Realizar las conexiones necesarias entre el 
   código desarrollado y la interfaz que permitan permitir le uso del simulador por parte del usuario.

\end{enumerate}

De forma paralela mientras se hacia el desarrollo de la máquina superescalar se creaba 
un pequeño \textit{wireframe} de la interfaz original. 

\bigskip
Y también de forma paralela, tras terminar el wireframe, se realizaba la migración 
de la documentación original.

