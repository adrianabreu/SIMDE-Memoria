%%%%%%%%%%%%%%%%%%%%%%%%%%%%%%%%%%%%%%%%%%%%%%%%%%%%%%%%%%%%%%%%%%%%%%%%%%%%%
% Chapter 1: Introducción 
%%%%%%%%%%%%%%%%%%%%%%%%%%%%%%%%%%%%%%%%%%%%%%%%%%%%%%%%%%%%%%%%%%%%%%%%%%%%%%%

%---------------------------------------------------------------------------------
\section{Introducción}
\label{1:sec:1}

En el área de arquitectura de computadores, es común repasar fundamentos teóricos
que tienen una base histórica y que resultan difíciles de comprender debido 
a que a pasar de que son las bases de los sistemas modernos, conllevan 
una gran abstracción respecto a las soluciones que se implementan en la actualidad.

\bigskip
En este trabajo, se hace hincapié en el aspecto del paralelismo a nivel de instrucción, 
un aparte fundamental de las computadoras modernas y que permitió una mejora exponencial
en el rendimiento.

\bigskip
Sin embargo, a pesar de que han pasado ya más de veinte años de los diseños iniciales,
este y muchos otros conceptos no resultan fáciles de enseñar en la docencia. A pesar de que 
todo es abstraíble a una serie de fundamentos simples, resulta muy difícil visualizar el 
paso de la \textit{segmentación} al paralelismo a nivel de instrucción.

\bigskip
En este contexto, un simulador permite ayudar a comprender, de manera visual y esquematizada
como funcionan estos mecanismos y actuar como un potente catalizador para la asimilación de los 
conceptos.


%---------------------------------------------------------------------------------
\section{Paralelismo a nivel de instrucción}
\label{1:sec:2}

El paralelismo a nivel de instrucción 

\subsection{Superescalar}


\subsection{VLIW}

Aunque no se desea entrar mucho en detalle en este aspecto, puesto que este trabajo de 
fin de grado se centra en la máquina superescalar, debe nombrarse la otra gran 
representante del paralelismo a nivel de instruccin, las máquinas \textit{Very Long Instruction Word}.

\bigskip
Mientras que las máquinas Superescalares hacen uso de un hardware complejo para llevar
a cabo el paralelismo a nivel de instrucción, las máquinas VLIW toman una vía opuesta.

\bigskip
El hardware se mantiene relativamente simple, y es el compilador, el que se encarga
de realizar las operaciones necesarias para permitir que se aproveche al máximo el 
paralelismo.

%---------------------------------------------------------------------------------
\section{Motivación para el trabajo}
\label{1:sec:3}

Como se ha mencionado, el uso de un simulador para apoyar la docencia de esta área
de Arquitecutra de Computadores resultaba un campo realmente interesante. De hecho,
esta herramienta ya existe. El actual profesor de la Universidad de La Laguna 
Iván Castilla desarrolló un simulador en C++ con este propósito.

\bigskip
Este simulador se ha estado utilizando como un complemento más de la docencia, pero
con el paso del tiempo, ha quedado obsoleto. No tanto por su funcionalidad, puesto
que los fundamentos teóricos sobre los que se basa no han cambiado con el tiempo, como 
por su aspecto visual y su accesibilidad.

\bigskip
Es por esto, se ha querido recuperar esta herramienta para continuar con su desarrollo 
y ampliación y este trabajo de fin de grado se centra en migrar esta aplicación a versión 
web de tal forma que sirva como base para los futuros proyectos.

