%%%%%%%%%%%%%%%%%%%%%%%%%%%%%%%%%%%%%%%%%%%%%%%%%%%%%%%%%%%%%%%%%%%%%%%%%%%%%%%
% Chapter 3: Tecnologías
%%%%%%%%%%%%%%%%%%%%%%%%%%%%%%%%%%%%%%%%%%%%%%%%%%%%%%%%%%%%%%%%%%%%%%%%%%%%%%%

%++++++++++++++++++++++++++++++++++++++++++++++++++++++++++++++++++++++++++++++


%++++++++++++++++++++++++++++++++++++++++++++++++++++++++++++++++++++++++++++++
\section{Typescript}
\label{3:sec1}

Podemos considerar a Typescript una pieza clave en el desarrollo del proyecto. 
A pesar de las distintas comodidades que han ido apareciendo en el mundo de Javascript con 
los distintos estándares,  (es2015, es2016). Sigue siendo complicado trabajar con programación 
orientada a objetos de forma natural en Javascript.  

\bigskip
Typescript es un lenguaje libre y de código abierto desarrollado por Microsoft que
 actúa como un superconjunto de javascript, y donde una de sus características más 
 destacables es la capacidad de añadir tipado estático.

\bigskip
Este tipado no se refleja en el código final, de hecho una interfaz, por ejemplo,
añade 0 sobrecarga en el código final. Pero si que es interesante por las capacidades de
autocompletar (a través de Microsoft Intellisense)  que añade. Typescript se puede transpilar 
directamente a código javascript es5, el cuál es el estándar actual en todos los navegadores.    

\bigskip
Si tuviéramos que barajar una posible alternativa, sin duda la más destacada sería Flow, 
un comprobador de tipos para javascript desarrollado por Facebook. 

\bigskip
Pero existen múltiples razones para que haya escogido typescript sobre Flow:

\begin{enumerate}
\item Flow no está siendo parte de un proyecto de gran envergadura, se suponía que
 se introduciría en el desarrollo de react. Sin embargo no ha sido así.

\item Typescript es la opción de defacto para Angular 2, ha ido ganando mucho entusiasmo
 por parte de la comunidad y tiene un gran soporte.

\item Tengo cierta experiencia con Typescript.

\end{enumerate}

%++++++++++++++++++++++++++++++++++++++++++++++++++++++++++++++++++++++++++++++
\section{React}
\label{3:sec2}

    Desde el inicio del proyecto, tenía claro que alguna librería se encargaría de realizar por mí el tedioso proceso de manipular el DOM. Actualmente existen múltiples librerías y frameworks que podían servirme en esta tarea, pero muchos de ellos (como por ejemplo Angular), no son lo suficientemente flexibles y acaban condicionando el desarrollo de la aplicación.

    Lo que SIMDE necesitaba era un proyecto que hiciera uso de los Web components, una serie de apis que permiten crear etiquetas html personalizadas y reutilizables. Los webs componentes consisten en 4 características principales que pueden usarse por separado o todos juntos:

Elementos personalizados
Shadow DOM
HTML Imports
HTML Templates

Probablemente la librería basada en Web Componentes más conocida es Polymer,una librería desarrollada por Google y anunciada en 2013. 
    Pero a pesar de que las múltiples ventajas de Polymer existe una librería cuya comunidad es espléndida, y no es otra que React.
    Existen una gran cantidad de motivos para escoger React sobre Polymer: COmo por ejemplo, que ahora mismo se utiliza en decenas de aplicaciones importantes, como netflix, airbnb, Wallmart…
    Que la comunidad es impresionantemente activa y cuenta con una gran cantidad de usuarios dispuestos a ayudar, así como cuenta con muchísima documentación y muchísimas implementaciones de librerías de terceros.
    React utiliza un híbrido entre html y javascript denominado jsx, como también tiene soporte para typescript, en este caso utilizamos tsx, y se basa en un “unidirectonial data-flow”. Ademas React implementa operaciones sobre el DOM virtual de tal forma que las operaciones sobre el verdadero DOM sean eficientes.
%++++++++++++++++++++++++++++++++++++++++++++++++++++++++++++++++++++++++++++++
\section{Webpack}
\label{3:sec3}

    Una de las grandes herramientas de 2016 que acabó por cambiar el flujo de muchos desarrolladores web y desbancó a tasks runners como Grunt y Gulp fue webpack.

    Para poder integrar todo este código y resolver el problema de los múltiples imports era necesario utilizar algún tipo de herramienta de gestión de paquetes, como por ejemplo commonjs, o requirejs. Sin embargo, webpack se encarga de resolver todas estas dependencias y crear statics assets para el navegador.



Uno de los mayores puntos a favor es que es altamente configurable, existen muchísimos plugins de webpack que permiten hacer preprocesamiento de css, tratamiento de imágenes, minimización codigo, étctera.. 

Actualmente en la aplicación de SIMDE webpack  se encarga de compilar el código de typescript, el código de react y de generar un bundle 