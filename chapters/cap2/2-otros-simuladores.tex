Sería irrazonable embarcarse en la tarea de migrar SIMDE sin comprobar antes si ya existe alguna
herramienta que se adapte a estas necesidades.

\begin{table}[ht]
\begin{center}
    \begin{tabular}{| l | p{12cm} |}
    \hline
    Simulador & Descripción \\ \hline
    SESC \cite{SESC} & Simulador de máquinas Superescalares desarrollado en la Universidad de Illinois, posee
    características tan interesantes como el SMT, el CMP... Pero carece de interfaz gráfica. \\ \hline
    ReSIM \cite{ReSIM} & Este simulador de máquinas Superescalares está basado en la ejecución por trazas. Se encuentra 
    disponible para múltiples dispositivos Xilink. \\ \hline
    SATSim \cite{SATSim} & Simulador de máquinas superescalares con traja que utiliza una simulación interactiva. \\ \hline
    VLIW-DLX \cite{VLIWDLX} & Basado en WinDLX y desarrollado en la Universidad de praga, intenta aportar un conocimiento 
    detallado de las máquinas VLIW. \\ \hline
    \end{tabular}
    \caption{Tabla comparativa de simuladores}
\end{center}
\end{table}

Como vemos, aunque existen múltiples simuladores dentro de esta sección, 
ninguno cumple con el objetivo inicial, muchos se centran más en estudiar los
efectos de utilizar un tipo concreto de arquitectura implementando máquinas 
específicas -incluso careciendo de interfaz-. Y el qué más se asemejaría a nuestro
propósito, el SATSim, tiene más de una década.