A modo de curiosidad, en el proyecto original, se consideraba poco factible la realización de 
un simulador de estas características en el entorno debido al escaso rendimiento. Cito textualmente:

\begin{quotation}
Utilizar programación web. Pese a las muchas ventajas de la programación
web en aspectos como la distribución y accesibilidad del programa, la realidad
es que el uso que se iba a dar a esta herramienta no invitaba a la utilización de
esta opción. El simulador no estaba planteado como una herramienta con
arquitectura cliente-servidor, ni tampoco se pretendía una ejecución distribuida.
La idea era que se procesara en la máquina del alumno y no en un servidor. El
uso de applets de java tampoco era una opción atractiva debido a la lentitud de
ejecución de java por ser interpretado.
\end{quotation} \cite{SIMDE}

\bigskip
Por sorprendete que resulte, este argumento totalmente justificado y válido en el momento
de su enunciado, resulta difícil de defender en un contexto actual. Hoy en día,
el rendimiento del lenguaje Javascript en el navegador es increíblemente alto. Y si valoramos que 
la diferencia de edad entre el documento citado y éste mismo es de unos escasos trece años -aunque 
trece años en el mundo de la tecnología no es un lapso de tiempo despreciable- se puede observar
que la diferencia es asombrosa.
 
\bigskip
Si intentamos encontrar un origen para esta diferencia tan increíble de rendimiento, sin duda
es necesario remontarse a la aparición de las primeras grandes aplicaciones que hacen uso de la tecnología
\textit{Ajax} como por ejemplo Gmail.\cite{EvolutionJavascript}

\bigskip
La tendencia a diseñar más y más aplicaciones utilizando esta tecnología resultó en una guerra por aumentar el rendimiento por parte de los distintos navegadores. Que ha 
acabado con el escenario actual, donde no sólo cada vez más aplicaciones de escritorio tienen una versión web 
disponible (véase las herramientas de Microsoft Office), si no que además existen muchos frameworks para 
desarrollar aplicaciones de escritorio a partir de aplicaciones web como \textit{Electron}.\cite{Electron}
