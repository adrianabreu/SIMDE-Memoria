Para integrar la documentación en la nueva aplicación web de SIMDE resultaba obvio que esta documentación
estuviera también en formato web. Para esto existían muchas alternativas, desde un conjunto de ficheros
html hasta un pequeño sistema de gestión de contenidos. 

\bigskip
Dado que la documentación es bastante extensa pero que en realidad, no es más que un documento 
que se redactará en una ocasión y se le irán realizando pequeñas ampliaciones y/o correcciones
se optó por una solución diferente, los generadores de contenido estático.

\subsection{Generadores de contenido estático}

Los generadores de contenido estático se encargan -resumido de forma breve- de generar 
un conjunto de ficheros \textit{HTML} y \textit{CSS} a partir de una plantilla y una serie 
de ficheros fuente, basándose normalmente en un formato de entrada común. Actualmente
este formato es el \textit{Markdown} \cite{GeneradoresEstaticos}. 

\bigskip
Este tipo de generadores estáticos tiene un gran auge entre los desarrolladores que desean 
mantener un blog, sin tener que estar desplegando bases de datos o teniendo que mantene múltiples
ficheros html. 

\bigskip 
Existen múltiples ventajas de utilizar este tipo de tecnologías, una de las más importantes
 es que se alimentan de un formato como es el markdown. El cual es realmente intuitivo 
 y tiene una amplia acogida más allá de este tipo de tecnologías. 

\subsection{Hexo}

Hexo es un generador de contenido estático basado en NodeJS \cite{Hexo}. 
No posee demasiadas diferencias destacables sobre el resto de alternativas,
y ha sido escogido para este proyecto principalmente porque al estar
basado en Javascript todo queda enfocado hacia un mismo ecosistema dando una 
sensación de homogeneidad con el resto del proyecto.