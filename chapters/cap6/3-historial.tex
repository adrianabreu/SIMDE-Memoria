Otra característica que resultaba necesaria en el simulador y que se añoraba sobre todo en el primer
contacto era la posibilidad de ir hacia atrás.

\bigskip
En un principio, se esperaba utilizar algún sistema gestor de estados y permitir el time traveling. 
Por desgracia debido al volumen de este trabajo de fin de grado decidió no utilizarse este tipo de 
soluciones.

\bigskip
Sin embargo, la idea del \textit{time traveling} resultaba más que deseable, por lo que se implementó
de una forma un tanto rústica. Para permitir emular este comportamiento y mantener la fidelidad de 
la ejecución (recordemos que existen una serie de operaciones que son sujetas a un porcentaje de fallos
aleatorios), lo que se hizo fue acumular el estado visual de la máquina, el modelo en sí con el 
que se dibujaban los componentes.

\bigskip
De tal forma, cuando un usuario entraba en este modo timetraveling recorre un array de estados de la 
interfaz, imprimiendo la información almacenada, sin afectar al comportamiento de la máquina, pero 
emulando ese \textit{time traveling} de cara al usuario.

\bigskip
Por tanto, se considera que un usuario que retroceda X pasos, deberá avanzar esos X pasos para continuar
la ejecución (o en su defecto, pulsar el botón \textbf{Play}). Esto es así porque aunque como repito, se 
trata de una característica enormemente deseada en la primera toma de contacto del simulador, no se trata
de una característica de la que se espere que el usuario abuse.