Otra característica que resultaba necesaria en el simulador y que se añoraba sobre todo en el primer
contacto era la posibilidad de ir hacia atrás.

\bigskip
En un principio, se esperaba utilizar algún sistema gestor de estados y permitir el time traveling. 
Por desgracia debido al volumen de este trabajo de fin de grado decidió no utilizarse este tipo de 
soluciones.

\bigskip
Sin embargo, la idea del \textit{time traveling} resultaba más que deseable, por lo que se implementó
una versión más simple de la misma pero con la misma funcionalidad. Para permitir emular este 
comportamiento y mantener la fidelidad de la ejecución (es importante recordar que existen una serie de operaciones que están sujetas a un porcentaje de fallos
aleatorios), lo que se hizo fue acumular el estado visual de la máquina, el contenido en sí que muestran
los componentes.

\bigskip
De esta forma cuando un usuario entra en este modo de \textit{time traveling} lo que hace es
 recorrer un array de estados de la interfaz, imprimiendo la información almacenada sin afectar 
 al comportamiento de la máquina, pero representando esa sensación de retroceder y avanzar
 en la ejecución de cara al usuario.

\bigskip
Por tanto, se considera que un usuario que retroceda X pasos, deberá avanzar esos X pasos para continuar
la ejecución (o en su defecto, pulsar el botón \textbf{Play}).