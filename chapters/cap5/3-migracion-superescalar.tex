En un principio se analizó el código de la versión original de SIMDE\cite{SIMDE} y, utlizando como
referencia la memoria del mismo, donde se incluían las decisiones de diseño
tomadas para la realización del simulador, se concluyó que era una buena base para la realización 
de este simulador.

\bigskip
Aún así, en la migración de C++ a Typescript se echaron en falta múltiples características del primer lenguaje,
como por ejemplo: las estructuras, la sobrecarga de operadores y el uso de iteradores sobre colecciones.

\bigskip
La mayor dificultad en esta parte del proceso era ser capaz de seguir el flujo de un volumen tan grande
de código. Por suerte, la \textit{linterna} de Typescript y el \textit{intellisense} ayudaron mucho a reducir la cantidad
de errores sintácticos.

\bigskip
Aún así era notorio el hecho de que el código tenía más de una década. Al usarse C++98 muchás de las librerías
no estaban incluidas en el éstandar, lo cual dificultaba entender este proceso. Por ejemplo, la siguiente
porción de código dió bastantes problemas en un primer momento.

\begin{lstlisting}[language=C++]
TStringList *etiquetas; // Lista de etiquetas
if ((ind = etiquetas->IndexOf(str + AnsiString(":"))) != -1)
   bbas = (bBasico *)etiquetas->Objects[ind];
\end{lstlisting}

\bigskip
Este bloque resultó especialmente confuso al castear un elemento de una lista de strings 
a una estructura del tipo bloque básico. Por suerte, se pudo encontrar una referencia 
a la clase \textit{TStringList} en internet, explicando que cada uno de los elementos de la lista
eran de la clase \textit{TString} la cual, tiene una propiedad que representa un vecotr de objetos
asociado \cite{TStringObjects} .

\bigskip
Por comodidad durante el desarrollo, se convirtieron las estructuras de C++ en clases, de tal forma
que la gestión de las mismas fuera más sencilla.

\bigskip
Se aplicaron algunos tests para la clase de \textit{CircularBuffer}, que representaba una parte importante
del funcionamiento del prefetch y el decoder y cuyo mal uso acarreó varios errores durante el desarrollo.