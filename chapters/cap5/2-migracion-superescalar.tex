Una de las cosas que más puedo agradecer, es que el diseño del autor original de SIMDE era bastante 
bueno. Además, en la memoria del proyecto \textbf{CITAR MEMORIA} se incluían las decisiones de diseño
tomadas para la realización del simulador. Dando sentido.

\bigskip
Aún así, en la migración de C++ a Typescript se echaron múltiples características del primer lenguaje,
como por ejemplo: las estructuras,  la sobrecarga de operadores y el uso de iteradores sobre colecciones.

\bigskip
La mayor dificultad en esta parte del proceso era ser capaz de seguir el flujo de un volumen tan grande
de código. Por suerte, la linterna de Typescript y el \textit{intellisense} ayudaron mucho a reducir la cantidad
de pifias.

\bigskip
Aún así encontré además, algunos problemas a la hora de migrar el código debido al uso de librerías 
que no estaban en el estándar.

\textbf{CITAR FAMOSO CASO BBAS}.

\bigskip
Por comodidad durante el desarrollo, se convirtieron las estructuras de C++ en clases, de tal forma
que la gestión de las mismas fuero más sencilla.