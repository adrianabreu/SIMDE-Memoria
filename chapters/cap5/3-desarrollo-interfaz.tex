Esta tarea, aunque en un principio pudiera parecer mucho menos intensa que la migración del código 
superescalar, también ha tenido una carga de trabajo. Todo ello porque se ha intentado por encima 
de todo, conseguir un alto grado de modularización y reutilización. 

\subsection{Análisis de la interfaz original}

Si analizamos la interfaz original de SIMDE nos encontramos con esto: 

A grosso modo podríamos diferenciar 5 \textit{"zonas"} principales.:

\begin{itemize}

\item La barra de herramientas.

\item La barra de accesos.

\item La zona del código.

\item La zona de ejecución.

\item La zona de memoria / registros. 

\end{itemize}

Se ha considerado que lo más sensato es agrupar las ventanas 

Ahora analizaremos el funcionamiento de los componentes, en sí, con el objetivo de aislar conceptos
/funcionalidades.

\subsection{El nuevo diseño web}

A día de hoy 

\bigskip
Ahora surge un problema, ¿merece el esfuerzo aplicar el diseño de las ventanas redimensionables y 
arrastables? La conclusión es que no. SIMDE muestra la información que necesitamos. En un 
principio nos interesa ver la ejecución en sí. Luego, en su realización, nos interesa ver simplemente
el resultado de la ejecución.

\bigskip
Por eso, se ha decidido separar la ejecución de los registros / memoria mediante el uso de pestañas.

a) Pestaña 1.

b) Pestaña 2.

\subsection{El nuevo diseño por componentes}