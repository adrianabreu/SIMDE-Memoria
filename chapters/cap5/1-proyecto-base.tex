Para llevar a cabo este proyecto se ha decidido trabajar con un proyecto de NodeJS que gestionara
las dependencias. Se ha utilizado como base el repositorio de github \textit{typescript-starter}.\cite{TypescriptStarter}
Este proyecto incluía las dependencias necesarías para empezar a trabajar en la lógica, como 
el compilador de Typescript y el \textit{test runner} Ava.

\bigskip
Para gestionar las dependencias se ha utilizado Yarn\cite{yarn}. Este gestor de dependencias surgió
como alternativa a npm\cite{npm}. Al igual que todos los proyectos basados en NodeJS, el fichero
más importante es el \textit{package.json}, donde se definen las dependencias y las acciones.

\bigskip
Para instalar las dependencias basta con ubicarse en directorio raíz del proyecto y ejecutar el siguiente
comando.

\begin{lstlisting}
yarn install
\end{lstlisting}

\bigskip
Si se quiere continuar desarrollando el proyecto, debe utilizarse el comando: 
\begin{lstlisting}
yarn start
\end{lstlisting}

Este comando arrancará un servidor embebido en webpack con BrowserSync, de tal forma que se la 
página se recargará automáticamente cada vez que se modifican los ficheros, 
permitiendo ver los cambios realizados.

\bigskip
Si se quiere publicar la aplicación debe ejecutarse el comando:
\begin{lstlisting}
yarn run webpack:prod
\end{lstlisting}

De esta forma se activa el perfil de producción de webpack, se minimiza el código y se generan 
los ficheros estáticos a publicar en la página web en la carpeta \textit{dist}.
