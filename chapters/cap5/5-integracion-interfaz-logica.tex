Una de las mayores ventajas que aporta la flexibilidad de React es que la única limitación 
real para el correcto funcionamiento de la librería consiste en que el código javascript 
en sí no debe realizar modificaciones sobre los elementos propios de los componentes de React.

\bigskip
Para realizar la integración entre la lógica y la interfaz se ha recurrido al uso de los callbacks.
La idea es utilizar un sistema de gestión de estados que cuelgue del objeto del navegador \textit{window}.
Cuando un componente se renderiza, se suscribe a un objeto general que concentra todos los componentes
y los pasos o "tics" se invocan mediante el uso de los callbacks.

\bigskip
Dado que todos los componentes creados para la interfaz hacían uso de estos callback se ha aplicado la 
herencia. Todos los componentes extienden la clase \textit{BaseComponent} en vez de la clase 
\textit{React.Component}.

\bigskip
Además, utilizando el mismo \textit{título} que tenga ese componente, se decide mediante un simple switch
cual es el contenido que lleva asociado. Esta abstracción permitirá extender este modelo en un futuro 
para la máquina VLIW.