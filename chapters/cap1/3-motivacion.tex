Como se ha mencionado, el uso de un simulador para apoyar la docencia de esta área
de Arquitectura de Computadores resultaba un campo realmente interesante y relevante. 
De hecho, un simulador que cumpla estas características, ya existe. El profesor 
de la Universidad de La Laguna Iván Castilla desarrolló un simulador en C++ con este propósito.

\bigskip
Este simulador se ha estado utilizando como un complemento fundamental en la docencia
de Arquitectura de Computadores, pero con el paso del tiempo, ha quedado obsoleto. 
No tanto por su funcionalidad, puesto que los fundamentos teóricos sobre los que 
se basa no han cambiado con el tiempo, como por su aspecto visual y su accesibilidad.

\bigskip
Es por esto se ha querido recuperar esta herramienta para continuar con su desarrollo 
y ampliación y este trabajo de fin de grado se centra en migrar esta aplicación a versión 
web de tal forma que sirva como base para los futuros proyectos.