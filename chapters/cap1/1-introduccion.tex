El peso de la tecnología en la eficiencia de los computadores actuales
es innegable. Sin embargo, los conceptos básicos que definen la 
arquitectura de estos computadores se basan en ideas de hace varias décadas.
En el ámbito docente resulta difícil transmitir estos fundamentos, ya que 
han quedado superpuestos por el aumento de la complejidad de la arquitectura,
haciendo que sea imprescindible una gran abstracción por parte del alumno 
al no poder disponer de las máquinas originales en los que se implementaron.

\bigskip
Las herramientas docentes típicas, como pueden ser una pizarra, un libro 
de texto o diapositivas, tienen una capacidad limitada para representar 
los fundamentos ya expuestos.

\bigskip
Este trabajo se centra en el uso de simuladores como medio de apoyo docente
para el tema del  paralelismo a nivel de instrucción, una parte fundamental
en el incremento de rendimiento de las computadoras.

\bigskip
En este contexto, los simuladores juegan una pieza clave en el campo de 
la Arquitectura de Computadores, permitiendo asociar fundamentos y teorías,
 simplificando abstracciones y facilitando la labor docente.