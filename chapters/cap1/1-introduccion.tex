En el área de arquitectura de computadores, es común repasar fundamentos teóricos
que tienen una base histórica y que resultan difíciles de comprender debido 
a que a pasar de que son las bases de los sistemas modernos, conllevan 
una gran abstracción respecto a las soluciones que se implementan en la actualidad.

\bigskip
En este trabajo, se hace hincapié en el aspecto del paralelismo a nivel de instrucción, 
un aparte fundamental de las computadoras modernas y que permitió una mejora exponencial
en el rendimiento.

\bigskip
Sin embargo, a pesar de que han pasado ya más de veinte años de los diseños iniciales,
este y muchos otros conceptos no resultan fáciles de enseñar en la docencia. A pesar de que 
todo es abstraíble a una serie de fundamentos simples, resulta muy difícil visualizar el 
paso de la \textit{segmentación} al paralelismo a nivel de instrucción.

\bigskip
En este contexto, un simulador permite ayudar a comprender, de manera visual y esquematizada
como funcionan estos mecanismos y actuar como un potente catalizador para la asimilación de los 
conceptos.