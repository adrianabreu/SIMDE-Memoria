En el área de arquitectura de computadores, es común trabajar sobre fundamentos teóricos
que tienen una base histórica y que resultan difíciles de asimilar debido 
al desfase que existe con las implementaciones de los sistemas modernos. Es decir, estos 
fundamenos conllevan una gran abstracción para el alumno al no poder hacer un 
equivalente con las soluciones modernas.

\bigskip
Este trabajo se centra en el uso de simuladores como medio de apoyo docente para el 
tema del  paralelismo a nivel de instrucción, una parte fundamental en el incremento
de rendimiento de las computadoras.

\bigskip
Sin embargo, a pesar de que han pasado ya más de veinte años de los diseños iniciales de esta 
técnica, aún no existe una herramienta clave que ayude a transmitir los fundamentos de 
esta.

\bigskip
En este contexto, los simuladores juegan una pieza clave en el campo de la arquitectura de 
computadores, permitiendo asociar fundamenos y teorias, simplificando abstracciones y 
favorenciendo el entorno docente.