En el área de Arquitectura de Computadores, es común trabajar sobre fundamentos teóricos
que tienen una base histórica y a pesar de todo el tiempo que ha pasado desde 
que aparecieron estos conceptos y de la abundante bibliografía de la que se
dispone, resultan conceptos difíciles de asimilar debido en parte, al desfase que existe
con las implementaciones de los sistemas modernos. Es decir, estos 
fundamenos conllevan una gran abstracción para el alumno al no disponer de las máquinas originales
en las que se implementaron.

\bigskip
Este trabajo se centra en el uso de simuladores como medio de apoyo docente para el 
tema del  paralelismo a nivel de instrucción, una parte fundamental en el incremento
de rendimiento de las computadoras.

\bigskip
Sin embargo, a pesar de que han pasado ya más de cuarentas años de los diseños iniciales de esta 
técnica, aún no resulta sencillo explicar en detalle el funcionamiento de la misma.

\bigskip
En este contexto, los simuladores juegan una pieza clave en el campo de la Arquitectura de Computadores,
permitiendo asociar fundamentos y teorías, simplificando abstracciones y facilitando la labor docente.