En el área de Arquitectura de Computadores, es común trabajar sobre fundamentos teóricos
de hace varias décadas y a pesar de todo el tiempo que ha pasado desde 
que aparecieron y de la abundante bibliografía de la que se
dispone, resultan conceptos difíciles de asimilar debido en parte, al desfase que existe
con las implementaciones de los sistemas modernos. Es decir, los fundamentos teóricos
han quedado superpuestos por el aumento de la complejidad de la arquitectura, con lo que la 
asimiliación de estos fundamentos requiere de una gran abstracción por parde  del alumno al 
no disponer de las máquinas originales en las que se implementaron.

\bigskip
Las herramientas docentes típicas, como pueden ser una pizarra, un libro de texto
o diapositivas, tienen una capacidad limitada para representar los fundamentos
ya expuestos.

\bigskip
Este trabajo se centra en el uso de simuladores como medio de apoyo docente para el 
tema del  paralelismo a nivel de instrucción, una parte fundamental en el incremento
de rendimiento de las computadoras.

\bigskip
En este contexto, los simuladores juegan una pieza clave en el campo de la Arquitectura de Computadores,
permitiendo asociar fundamentos y teorías, simplificando abstracciones y facilitando la labor docente.