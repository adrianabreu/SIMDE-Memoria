Con el objetivo de mejorar el rendimiento de una computadora es necesario superar la barrera
de 1 CPI \textit{(ciclo por instrucción)}, para ello, se debe aumentar el grado de emisión de 
instrucciones.

\bigskip
Para alcanzar este propósito se deben considerar varios factores:

\begin{enumerate}

\item Se debe proveer de una estructura de emisión.

\item Se deben detectar los posibles riesgos asociados. 

\item Se debe hacer un \textit{scheduling}.

\end{enumerate}

\bigskip
Existen dos grandes técnicas para conseguir explotar este paralelismo a nivel de instrucción:

\begin{enumerate}

\item \textbf{Planificación dinámica}

\item \textbf{Planificación estática}

\end{enumerate}

\subsection{Superescalar}

Las máquinas Superescalares hacen uso de la planificación dinámica, es decir, el hardware 
en si es el encargado de mantener la emisión de múltiples instrucciones, decodificarlas y 
ejecutarlas.

\bigskip
Este tipo de parlaleismo conlleva un enorme grado de complejidad, ya que se deben abordar 
todos estos problemas:

\begin{enumerate}

\item Capturar y decodificar más instrucciones en un ciclo.

\item Detectar dependencias entre las instrucciones capturas y las que se ejecutan.

\end{enumerate}

\bigskip
Para resolver los problemas de dependencias se utiliza el algoritmo de Tomasulo.

CITAR ALGORITMO TOMASULO 

\subsection{VLIW}

A diferencia de las máquinas superescalares, las máquinas \textit{Very Long Instruction Word},
mantienen un hardware simple de emisión y todas las responsabilidades caen en el compilador.

\bigskip
Para mantener este hardware simple, estas máquinas trabajan como su propio nombre indica, con
un tamaño de palabra muy grande, es decir, las instrucciones de la máquina son un conjunto
de instrucciones normales que pueden ejecutarse en paralelo.

\bigskip
Esto por supuesto, implica una serie de ventaja / inconvenientes. Por una parte, el compilador,
debe tener una visión completa del programa, conlleva un nivel de complejidad elevado y requiere
un alto grado de especialización por máquina para poder realizar las mayores optimizacions posibles.

\bigskip
Por otra parte, sin embargo, el software es versátil y la circuitería al ser de una menor complejidad 
permite mayores velocidades de reloj.