Para aumentar el rendimiento de los computadores es necesario 
que se puedan ejecutar varias instrucciones en paralelo.

\bigskip
Desde 1985 la mayoría de computadoras han explotado el paralelismo
a nivel de instruccion implementando la técnica conocida como 
segmentación -una técnica que promueve una disposición concreta 
de los recursos de la máquina y una divsión de la ejecución de 
las instrucciones en etapas para conseguir un funcionamiento 
similar al de una cadena de montaje-. Esta técnica tiene como
límite máximo  el rendimiento de 1 ciclo por instrucción (CPI). 
Para poder superar este límite, es necesario emitir varias 
instrucciones diferentes de forma simultánea.\cite{Hennessy:2011}

\bigskip
Para alcanzar este propósito se deben considerar los posibles
riesgos que ello implica, al ejecutar multiples instrucciones
aparecen diversos riesgos estructurales y dependencias de 
distintos tipos: control, datos o recursos. Estas dependencias
 deben ser detectadas y resuelas correctamente.

\bigskip
Existen dos grandes técnicas para conseguir explotar el 
paralelismo a nivel de instrucción con emisión múltiple:

\begin{enumerate}

\item \textbf{Planificación dinámica}: La responsabilidad de
detectar y resolver los problemas recae en el propio hardware.

\item \textbf{Planificación estática}: Delega las decisiones en
 el software. Manteniendo un hardware relativamente simple.

\end{enumerate}

\subsection{Máquinas Superscalares}

Las máquinas superescalares son aquellas máquinas que cumplen 
las siguientes características:

\begin{itemize}

\item Emiten instrucciones desde un único flujo secuencial de 
instrucciones.

\item Poseen emisión múltiple.

\item Hacen uso de la planificación dinámica.

\end{itemize}

\bigskip
Para resolver los problemas de dependencias se utiliza una 
modificación del algoritmo de Tomasulo. Este algoritmo nació
originnalmente con el objetivo de controlar las dependencias en 
la unidad funcional de punto flotante de la máquina 360/91 de IBM.

\bigskip
Mediante un seguimiento de los operandos y renombrado de 
registros se eliminaron los tres tipos de riesgos de datos:
\textit{Read After Write}, \textit{Write After Read} 
y \textit{Write After Write}. 

\bigskip
Además, las máquinas superescalares incorporan nuevas 
estructuras de hardware como el ReorderBuffer para permitir
la ejecución fuera de orden.