Para aumentar el rendimiento de los computadores es necesario que se puedan ejecutar varias 
instrucciones en paralelo.

\bigskip
Desde 1985 la mayoría de computadoras han explotado el paralelismo a nivel de instruccion implementando
la técnica conocida como segmentación -una técnica que promueve una disposición concreta de los 
recursos de la máquina y una divsión de la ejecución de las instrucciones en etapas para conseguir 
un funcionamiento similar al de una cadena de montaje-. 
Esta técnica tiene como límite máximo  el rendimiento de 1 ciclo por instrucción (CPI). 
Para poder superar este límite, es necesario emitir varias instrucciones diferentes de 
forma simultánea.\cite{Hennessy:2011}

\bigskip
Para alcanzar este propósito se deben considerar los posibles riesgos que ello implica, al ejecutar multiples
instrucciones aparecen diversos riesgos estructurales y dependencias de distintos tipos: Control, datos o recursos. Estas 
dependencias deben ser detectadas y resuelas correctamente.

\bigskip
Existen dos grandes técnicas para conseguir explotar el paralelismo a nivel de instrucción con emisión múltiple:

\begin{enumerate}

\item \textbf{Planificación dinámica}: La responsabilidad de detectar y resolver los problemas recae
en el propio hardware.

\item \textbf{Planificación estática}: Delega las decisiones en el software. Manteniendo un hardware
relativamente simple.

\end{enumerate}

Las máquinas superescalares son aquellas máquinas que cumplen las siguientes características:

\begin{itemize}

\item Emiten instrucciones desde un único flujo secuencial de instrucciones.

\item Poseen emisión múltiple.

\item Hacen uso de la planificación dinámica.

\end{itemize}

\bigskip
Para resolver los problemas de dependencias se utiliza una modificación del algoritmo de Tomasulo.
Este algoritmo nació originnalmente con el objetivo de controlar las dependencias en la unidad
funcional de punto flotante de la máquina 360/91 de IBM.

\bigskip
Mediante un seguimiento de los operandos y renombrado de registros se eliminaron los tres tipos 
de riesgos de datos: \textit{Read After Write}, \textit{Write After Read} y \textit{Write After Write}. 

\bigskip
Además, las máquinas superescalares incorporan nuevas estructuras de hardware como el ReorderBuffer para
permitir la ejecución fuera de orden.

\subsection{VLIW}

A diferencia de las máquinas Superescalares, las máquinas \textit{Very Long Instruction Word},
mantienen un hardware simple de emisión y todas las responsabilidades caen en el compilador.

\bigskip
Para mantener este hardware simple, estas máquinas trabajan como su propio nombre indica, con
un tamaño de palabra muy grande, es decir, las instrucciones de la máquina son un "grupo"
de instrucciones normales que pueden ejecutarse en paralelo.

\bigskip
Esto por supuesto, implica una serie de ventaja / inconvenientes. Por una parte, el compilador,
debe tener una visión completa del programa, conlleva un nivel de complejidad elevado y requiere
un alto grado de especialización por máquina para poder realizar las mayores optimizacions posibles.

\bigskip
Por otra parte, sin embargo, el software es versátil y la circuitería al ser de una menor complejidad 
permite mayores velocidades de reloj.