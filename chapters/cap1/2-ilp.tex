Con el objetivo de mejorar el rendimiento de una computadora es necesario superar la barrera
de 1 ciclo por instrucción -el límite teórico máximo que se puede conseguir con la técnica de segmentación-,
para ello, se debe conseguir ejecutar varias instrucciones diferentes de forma paralela.\cite{Hennessy:2011}

\bigskip
Para alcanzar este propósito se deben considerar varios factores:

\begin{itemize}

\item Se debe proveer de una estructura capaz de emitir múltiples instrucciones.

\item Se deben detectar los posibles riesgos asociados (tanto entre las instrucciones que se van a 
emitir entre sí como entre las que se van a emitir y las que ya están ejecutándose). 

\item Se debe hacer una planificación para evitar que sucedan los posibles riesgos de datos o 
de estructuras, y también optimizar el funcionando de la máquina.

\end{itemize}

\bigskip
Existen dos grandes técnicas para conseguir explotar el paralelismo a nivel de instrucción:

\begin{enumerate}

\item \textbf{Planificación dinámica}: La responsabilidad de detectar y resolver los problemas recae
en el propio hardware.

\item \textbf{Planificación estática}: Delega las decisiones en el software. Manteniendo un hardware
relativamente simple.

\end{enumerate}

\subsection{Superescalar}

La mayoría de máquinas Superescalares hacen uso de la planificación dinámica, es decir, el hardware 
en sí mismo es el encargado de mantener la emisión de múltiples instrucciones, decodificarlas y 
ejecutarlas.

\bigskip
Para resolver los problemas de dependencias se utiliza una modificación del algoritmo de Tomasulo.
Este algoritmo nació originnalmente con el objetivo de controlar las dependencias en la unidad
funcional de punto flotante de la máquina 360/91 de IBM.

\bigskip
Mediante un seguimiento de los operandos y renombrado de registros se eliminaron los tres tipos 
de dependencias \textit{Read After Write}, \textit{Write After Read} y \textit{Write After Write}. 

\bigskip
Además, las máquinas superescalares incorporan nuevas estructuras de hardware como el ReorderBuffer para
permitir la ejecución fuera de orden.

\subsection{VLIW}

A diferencia de las máquinas Superescalares, las máquinas \textit{Very Long Instruction Word},
mantienen un hardware simple de emisión y todas las responsabilidades caen en el compilador.

\bigskip
Para mantener este hardware simple, estas máquinas trabajan como su propio nombre indica, con
un tamaño de palabra muy grande, es decir, las instrucciones de la máquina son un "grupo"
de instrucciones normales que pueden ejecutarse en paralelo.

\bigskip
Esto por supuesto, implica una serie de ventaja / inconvenientes. Por una parte, el compilador,
debe tener una visión completa del programa, conlleva un nivel de complejidad elevado y requiere
un alto grado de especialización por máquina para poder realizar las mayores optimizacions posibles.

\bigskip
Por otra parte, sin embargo, el software es versátil y la circuitería al ser de una menor complejidad 
permite mayores velocidades de reloj.