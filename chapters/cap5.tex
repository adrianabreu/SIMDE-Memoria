%%%%%%%%%%%%%%%%%%%%%%%%%%%%%%%%%%%%%%%%%%%%%%%%%%%%%%%%%%%%%%%%%%%%%%%%%%%%%
% Chapter 5: Desarrollo del proyecto
%%%%%%%%%%%%%%%%%%%%%%%%%%%%%%%%%%%%%%%%%%%%%%%%%%%%%%%%%%%%%%%%%%%%%%%%%%%%%%%

%++++++++++++++++++++++++++++++++++++++++++++++++++++++++++++++++++++++++++++++

\section{Migración del núcleo de la aplicación}
\label{5:sec1} 

\subsection{Analizadores léxicos}
El núcleo de la aplicación se basa en el uso del generador de analizadores 
léxicos FLEX para parsear un conjunto de instrucciones similar a las del MIPS IV. 
Para realizar el proceso de migración he tenido que comprender primero el 
funcionamiento de los analizadores léxicos.

\bigskip
Un analizador léxico es un programa que recibe como entrada el código fuente de 
otro programa y produce una salida compuesta de tokens o símbolos 
que alimentarán a un analizador sintáctico.

\bigskip
Para poroceder con esta tarea se ha aislado la implementación original y se han 
realizado pequeñas pruebas concretas.

\subsection{Lex}

A pesar de las multiples librerías, que hay disponibles, se decidió que la más adecuada 
para este proyecto era Lex https://www.npmjs.com/package/lex. 

El funcionamiento de este paquete es realmente sencillo, partiendo de un 
objeto Lexer, se definen las reglas y el token a retornar. De esta forma
el código original que alimentaba flex:

// INSERTAR CODIGO original

Se ha convertido en: 

// CODIGO FLEX

%------------------------------------------------------------------------------
\section{Migración de la máquina superescalar}
\label{5:sec2} 

Una de las cosas que más puedo agradecer, es que el diseño del autor original de SIMDE era bastante 
bueno. Además, en la memoria del proyecto \textbf{CITAR MEMORIA} se incluían las decisiones de diseño
tomadas para la realización del simulador. Dando sentido.

\bigskip
Aún así, en la migración de C++ a Typescript se echaron múltiples características del primer lenguaje,
como por ejemplo: las estructuras,  la sobrecarga de operadores y el uso de iteradores sobre colecciones.

\bigskip
La mayor dificultad en esta parte del proceso era ser capaz de seguir el flujo de un volumen tan grande
de código. Por suerte, la linterna de Typescript y el \textit{intellisense} ayudaron mucho a reducir la cantidad
de pifias.

\bigskip
Aún así encontré además, algunos problemas a la hora de migrar el código debido al uso de librerías 
que no estaban en el estándar.

\textbf{CITAR FAMOSO CASO BBAS}.

\bigskip
Por comodidad durante el desarrollo, se convirtieron las estructuras de C++ en clases, de tal forma
que la gestión de las mismas fuero más sencilla.

%------------------------------------------------------------------------------
\section{Desarrollo de la interfaz}
\label{5:sec3} 

Esta tarea, aunque en un principio pudiera parecer mucho menos intensa que la migración del código 
superescalar, también ha tenido una carga de trabajo. Todo ello porque se ha intentado por encima 
de todo, conseguir un alto grado de modularización y reutilización. 

\subsection{Análisis de la interfaz original}

Si analizamos la interfaz original de SIMDE nos encontramos con esto: 

A grosso modo podríamos diferenciar 5 zonas principales.

Ahora analizaremos el funcionamiento de los componentes, en sí, con el objetivo de aislar conceptos
/funcionalidades.

\subsection{El nuevo diseño web}

A día de hoy 

\bigskip
Ahora surge un problema, ¿merece el esfuerzo aplicar el diseño de las ventanas redimensionables y 
arrastables? La conclusión es que no. SIMDE muestra la información que necesitamos. En un 
principio nos interesa ver la ejecución en sí. Luego, en su realización, nos interesa ver simplemente
el resultado de la ejecución.

\bigskip
Por eso, se ha decidido separar la ejecución de los registros / memoria mediante el uso de pestañas.

a) Pestaña 1.

b) Pestaña 2.

\subsection{El nuevo diseño por componentes}

%------------------------------------------------------------------------------
\section{Integración interfaz - lógica desarrollada}
\label{5:sec4} 

\subsection{Realización de los bindings}
Uno de los puntos a favor de haber utilizado este tipo de librerías es que su elevada flexibilidad
nos otorga cierto grado de 

%------------------------------------------------------------------------------
\section{Migración de la documentación}
\label{5:sec5} 

La primer ampliación ha sido la incorporación de documentación del programa. Aunque el
término de ampliación no es del todo correcto, puesto que en el proyecto original el autor
elaboró una extensa documentación, esta quedó inaccesible.

\bigskip
La documentación fue realizada en formato .HLP, un formato de ayuda de Windows que quedó
en desuso en Windows Vista. Y esta documentación era realmente interesante, pues no sólo
contenía datos sobre la palicación, sino que incluía consejos para el desarrollo de 
aplicaciones para las distintas máquinas y además explicaba el funcionamiento de las máquinas.

\bigskip
Para recuperar esta documentación se ha utilizado una herramientas de extracción denominada
Help Decompiler. Esta herramienta de línea de comandos procesa los ficheros de ayuda de
Windows .HLP y genera un fichero de texto enriquecido con la documentación y en una carpeta
externa el contenido multimedia que incluye la misma.

\bigskip
Para poder llevar a cabo la tarea de la documentación de forma paralela se consideró que lo mejor era
hacer un proyecto aparte. Resultaba evidente que la documentación de una aplicación web debía de estar en la
web. 

\bigskip
Tras barajar algunas opciones, se optó por mover la documentación a un formato mantenible como 
es markdown. Y partiendo de esto se utilizó un generador de contenido estático basado en NodeJS (Hexo)
para convertir este markdown en web. Se desarrollo un tema simple y personalizado para la ayuda y se 
añadió la capacidad de cambiar entre inglés y español.

\bigskip
Este es el estado actual de la aplicación de la documentación.

    IMAGEN DOCUMENTACIÓN

\bigskip
Se puede acceder a ella desde el menú \textit{Ayuda > Documentación} en la nueva aplicación del SIMDE.
%++++++++++++++++++++++++++++++++++++++++++++++++++++++++++++++++++++++++++++++
