%%%%%%%%%%%%%%%%%%%%%%%%%%%%%%%%%%%%%%%%%%%%%%%%%%%%%%%%%%%%%%%%%%%%%%%%%%%%%
% Chapter 7: Conclusiones y Trabajos Futuros 
%%%%%%%%%%%%%%%%%%%%%%%%%%%%%%%%%%%%%%%%%%%%%%%%%%%%%%%%%%%%%%%%%%%%%%%%%%%%%%%
\section{Conclusiones}
\label{7:sec1}

Con el desarrollo de este trabajo se ha conseguido disponer una nueva versión
del simulador de paralelismo a nivel de instrucción SIMDE.

Se ha visto además que las múltiples herramientas y tecnologías disponibles hoy
día (los nuevos lenguajes transpilables a javascript, los web components, las 
herramientas para la distribución), permiten elaborar y diseñar con relativa
facilidad aplicaciones que se salen de la norma.

Es necesario recordad, que como en todo proceso de software, el desarrollo de una 
aplicación esta vivo, sujeto a cambios y que dado el impresionante ritmo al que
evoluciona el mundo de la web, quizás este trabajo sea un pequeño anexo al pie 
de página de lo que esta aplicación pueda representar.

%++++++++++++++++++++++++++++++++++++++++++++++++++++++++++++++++++++++++++++++
\section{Líneas futuras}
\label{7:sec2}

Tras el desarrollo de este trabajo se abren varias líneas futuras: 

\begin{itemize}

\item Implementación de la máquina VLIW: Esta línea no resulta sorprendente. Con el desarrollo de este 
trabajo de fin de grado también se han implementado las estructuras básicas que se comparte con la máquina 
VLIW. Esta línea de trabajo tiene la mayor prioridad, pues equipara la funcionalidad de la aplicación web de 
SIMDE a la aplicación original.

\item Realizar una mayor cantidad de test: En el mundo web no resulta sencillo realizar test 
para los distintos casos, sin embargo, la lógica que acompaña al simulador es un gran candidato a 
ser testeado. Con las bases asentadas en los tests realizados para la estructura de la cola y 
del parseador del código, se podría extender este funcionamiento a pequeñas simulaciones.

\item Realizar un tutorial de inicio: Ahora que SIMDE es una aplicación accesible para todo el mundo, sería
deseable que no supusiera una barrera para cualquier usuario que sintiera la necesidad de probar 

\item Implementar un sistema de gestión de estados: En la aplicación actual se ha hecho un sistema
 de estados rudimentario debido al volumen de trabajo de sistemas como Redux y a las dificultades 
 que entrañan los Observables que vienen en un sistema como Mobx. En líneas futuras este sistema
  rudimentario podría sustituirse por un sistema más robusto desarrollado por terceros.

\end{itemize}