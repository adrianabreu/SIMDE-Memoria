%%%%%%%%%%%%%%%%%%%%%%%%%%%%%%%%%%%%%%%%%%%%%%%%%%%%%%%%%%%%%%%%%%%%%%%%%%%%%
% Chapter 7: Conclusiones y Trabajos Futuros 
%%%%%%%%%%%%%%%%%%%%%%%%%%%%%%%%%%%%%%%%%%%%%%%%%%%%%%%%%%%%%%%%%%%%%%%%%%%%%%%
\section{Conclusiones}
\label{7:sec1}

Con el desarrollo de este trabajo se ha conseguido disponer de una nueva versión
del simulador de paralelismo a nivel de instrucción SIMDE \cite{NuevaURLSimde}. 

Se ha visto además que las múltiples herramientas y tecnologías disponibles hoy
día (los nuevos lenguajes transpilables a javascript, los web components, las 
herramientas para la distribución), permiten elaborar y diseñar con relativa
facilidad aplicaciones que se salen de la norma.

Es necesario recordar que, como en todo proceso de software, el desarrollo de una 
aplicación está vivo, sujeto a cambios y dado el impresionante ritmo al que evoluciona
el mundo web, es posible que en un futuro aparezcan alternativas que permitan mejorar 
la experiencia de este simulador.

%++++++++++++++++++++++++++++++++++++++++++++++++++++++++++++++++++++++++++++++
\section{Líneas futuras}
\label{7:sec2}

Tras el desarrollo de este trabajo se abren varias líneas futuras: 

\begin{itemize}

\item Implementación de la máquina VLIW: Con el desarrollo de este trabajo de fin de grado también se han implementado las estructuras básicas que se comparte con la máquina 
VLIW. Esta línea de trabajo tiene la mayor prioridad, pues equipara la funcionalidad de la aplicación web de 
SIMDE a la aplicación original.

\item Realizar una mayor cantidad de test: En el mundo web no resulta sencillo realizar test 
para los distintos casos, sin embargo, la lógica que acompaña al simulador es un gran candidato a 
ser testeado. Con las bases asentadas en los tests realizados para la estructura de la cola y 
del parseador del código, se podría extender este funcionamiento a pequeñas simulaciones.

\item Implementar un sistema de gestión de estados: En la aplicación actual se ha hecho un sistema
de estados simple debido al volumen de trabajo que requiere incorporar tecnologías
como Redux y a las dificultades que entrañan los Observables que vienen 
en un sistema como Mobx. En líneas futuras este sistema podría sustituirse por un 
sistema más robusto desarrollado por terceros.

\item Intregar tutoriales de funcionamiento: Ahora que SIMDE es una aplicación con un gran
grado de accesibilidad, la única barrera a la que se enfrentan sus usuarios es a la dificultad de
comprender lo que están visualizando y el objetivo en sí de la máquina. A pesar de que esto
se explica en la documentación la integración de pequeños tutoriales de funcionamiento acabaría con 
esta barrera inicial y fomentaría el uso a gente con un menor conocimiento específico del campo
de arquitectura de computadores.

\item Automatizar el sistema de ejercicios: También con el objetivo de fomentar la autonomía podría
resultar interesante automatizar el sistema de ejercicios, de esta forma, mediante el uso de alguna
tecnología en \textit{backend} se podría no solo entregar al alumno un problema a resolver sino 
comparar la solución que ha propuesto con algunas soluciones propuestas por los profesores de antemano
de tal forma que el alumno sea capaz de recibir una retroalimentación instántanea sobre su solución.

\item Incluir gamificación: Mediante la incorporación de dinámicas de juego, se podría fomentar la
competitividad entre los usuarios, premiando la creatividad para la resolución de problemas e instando
a los alumnos a comprender mejor las arquitecturas de las máquinas y sus ventajas y limitaciones para
obtener códigos que requieran menor tiempo de ejecución.

\item Permitir el desarrollo colaborativo: Dado que uno de los requisitos principales de un 
ingeniero informático es la capacidad de trabajar en equipo, una línea de desarrollo interesante
en este sentido sería sincronizar la ejecución de las máquinas entre varios miembros de un grupo
mediante el uso de websockets. Así pues, si además se incluyera alguna herramienta de comunicación
 como un chat, se lograría fomentar una buena practica entre los alumnos y se dotaría de cierto
 dinamismo al desarrollo.

\item Desarrollar más simuladores: Con el desarrollo de este simulador se asientan las bases para
el futuro desarrollo de múltiples simuladores que permitan enseñar de forma interactiva
diversos fundamentos de la arquitectura de computadores como por ejemplo la el paralelismo a nivel 
de hilo o la coherencia a nivel de cache en sistemas multiproceso.

\end{itemize}