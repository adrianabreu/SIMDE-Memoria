%%%%%%%%%%%%%%%%%%%%%%%%%%%%%%%%%%%%%%%%%%%%%%%%%%%%%%%%%%%%%%%%%%%%%%%%%%%%%
% Chapter 8: Summary and Conlusions
%%%%%%%%%%%%%%%%%%%%%%%%%%%%%%%%%%%%%%%%%%%%%%%%%%%%%%%%%%%%%%%%%%%%%%%%%%%%%%%

%---------------------------------------------------------------------------------
\section{Summary}
\label{8:sec:1}

With the development of this work we have now available a new versión of
the Instruction Level Paralelism simulator SIMDE.

\bigskip
It has also been seen that the many tools and technologies available today (the new languages 
for the web, the webcomponents and the building tools), allowing to develop 
and design with relative ease complex applications.

\bigskip
It is necessary to remember that, as in all software processes, the development of a
application is alive, subject to changes and given the impressive pace at which the web world evolves.
It is possible that in the future there will be alternatives to improve the experience of this simulator.

%---------------------------------------------------------------------------------
\section{Future work lines}
\label{8:sec:2}

After the development of this work, several futurel lines are opened: 

\begin{itemize}

\item Implementation of the Very Long Instruction Word machine: Based on the initial design,
the application is ready for evolve and get the same capabilities that original simulator.
This line has the greatest priority since it equates the functionality of SIMDE's web application to
the original application.

\item Make a greater amount of test: In the web world is not easy to test 
 the multiple cases, however this application is a great candidate to be tested. 
As the basis for this laid in the tests made for the Code and the Queue structure,
the test could be extended to small simulations.

\item Implement a state management system: In the current application a simple management state system
have been used due to the high amount of work required to incorporate technologies such as Redux and 
the difficulties involved in the use of Observables in a system like Mobx. In future lines this system could be
replaced by a more robust system developed by third parties.

\item Include operating tutorials: Now that SIMDE is an application with great accessibility,
the only barrier faced by its users is the difficulty of understandoing what they are 
visualizing and the objective of the simulator itself. Although this is explained in the 
documentation the integration of small tutorials would put an end to this initial barrier
and would encourage the use to people with a less specific knowledge of the field
of computer architecture.

\item Automate the exercise system: With the aim of promoting autonomy could be interesting 
to automate the exercise system. By using some technology in the \textit{backend}, the student
could be given not only a problem to solve but the possibility of compare the solution 
he has proposed with some solutions made by the teachers beforehand. With that the student
could receive an important feedback instantly.

\item Include gamification: By incorporating gaming dynamics,competitiveness among users would promoted. 
Rewarding creativity for problem solving and urging students to understand the architectures 
of the machines and their advantages and limitations in order orden to develop codes 
that require less execution time.

\item Allow collaborative development: Since one of the main requirements of a
computer engineer is the ability to work in a team, an interesting development line
in this sense would be to synchronize the execution of the machines between several members of a group
through the use of websockets. Thus, if also some communication tool as a chat is included,
it would be possible to promote good practises among the students and it would provide 
dynamism to development.

\item Develop more simulators: With the development of this simulator, the basis are laid for
the future development of multiple simulators that allow to teach interactively multiple computer architecture's foundations
such as level thread parallelism or cache coherence in multiprocess systems.

\end{itemize}