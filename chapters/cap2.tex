%%%%%%%%%%%%%%%%%%%%%%%%%%%%%%%%%%%%%%%%%%%%%%%%%%%%%%%%%%%%%%%%%%%%%%%%%%%%%%%
% Chapter 2: Antecedentes
%%%%%%%%%%%%%%%%%%%%%%%%%%%%%%%%%%%%%%%%%%%%%%%%%%%%%%%%%%%%%%%%%%%%%%%%%%%%%%%

%++++++++++++++++++++++++++++++++++++++++++++++++++++++++++++++++++++++++++++++

\section{SIMDE}
\label{2:sec1}

En el año dos mil cuatro, el por aquel entonces estudiante de esta universidad, 
Iván Castilla Rodríguez - y ahora tutor de este trabajo de fin de grado-, 
desarrolló como proyecto final de carrera un Simulador didáctico para la enseñanza 
de arquitectura de computadores, el cual fue bautizado como Simde. 

\bigskip
Este simulador como se ha comentado en el apartado 1.4 cumple con las características
deseadas y esperadas de un simulador para la docencia de este ámbito.

\bigskip
Sin embargo, esta herramienta ya se encuentra desfasada. No ha sido un proyecto en constante
evolución, fue hecha utilizando C++98 y C++ Builder y necesita un nuevo enfoque.

\begin{figure}[!th]
\begin{center}
\includegraphics[width=0.8\textwidth]{images/cap2/simdeoriginal.eps}
\caption{Emulador original de SIMDE}
\label{fig:Emulador original de SIMDE}
\end{center}
\end{figure}

A día de hoy, una de las mejores soluciones para implementar este tipo de simulador es la web.
Permitiendo la accesibilidad por un mayor número de usuarios, centralizando el acceso a la 
aplicación y sobre todo, explotando todas las nuevas herramientas que se han ido generando
con el paso del tiempo.

\section{Otros simuladores}
\label{2:sec2}

Sería irrazonable embarcarse en la tarea de migrar SIMDE sin comprobar antes si ya existe alguna
herramienta que se adapte a estas necesidades.

\section{Evolución del mundo web}
\label{2:sec3}

A modo de curiosidad, en el proyecto original, se consideraba poco factible la realización de 
este trabajo de fin de grado en versión web debido al escaso rendimiento que ofrecían las 
soluciones disponibles (tanto el uso de Applets como Javascript).

\bigskip
Hoy en día, disponemos de aplicaciones realmente complejas, con una elaborada funcionalidad 
gracias a la evolución tanto de los estándares 

