%%%%%%%%%%%%%%%%%%%%%%%%%%%%%%%%%%%%%%%%%%%%%%%%%%%%%%%%%%%%%%%%%%%%%%%%%%%%%
% Chapter 1: Introducción 
%%%%%%%%%%%%%%%%%%%%%%%%%%%%%%%%%%%%%%%%%%%%%%%%%%%%%%%%%%%%%%%%%%%%%%%%%%%%%%%

%---------------------------------------------------------------------------------
\section{Introducción}
\label{1:sec:1}
El peso de la tecnología en la eficiencia de los computadores actuales es innegable. Sin embargo, los 
conceptos básicos que definen la arquitectura de estos computadores se basan en ideas de hace varias décadas.
Sin embargo, en el ámbito docente, resulta difícil transmitir estos fundamentos, ya que han quedado superpuestos
por el aumento de la complejidad de la arquitectura, haciendo que sea imprescindible una gran abstracción por 
parte del alumno al no poder disponer de las máquinas originales en los que se implementaron.

\bigskip
Las herramientas docentes típicas, como pueden ser una pizarra, un libro de texto
o diapositivas, tienen una capacidad limitada para representar los fundamentos
ya expuestos.

\bigskip
Este trabajo se centra en el uso de simuladores como medio de apoyo docente para el 
tema del  paralelismo a nivel de instrucción, una parte fundamental en el incremento
de rendimiento de las computadoras.

\bigskip
En este contexto, los simuladores juegan una pieza clave en el campo de la Arquitectura de Computadores,
permitiendo asociar fundamentos y teorías, simplificando abstracciones y facilitando la labor docente.


%---------------------------------------------------------------------------------
\section{Paralelismo a nivel de instrucción}
\label{1:sec:2}

Con el objetivo de mejorar el rendimiento de una computadora es necesario superar la barrera
de 1 CPI \textit{(ciclo por instrucción)}, para ello, se debe aumentar el grado de emisión de 
instrucciones.

\bigskip
Para alcanzar este propósito se deben considerar varios factores:

\begin{enumerate}

\item Se debe proveer de una estructura de emisión.

\item Se deben detectar los posibles riesgos asociados. 

\item Se debe hacer un \textit{scheduling}.

\end{enumerate}

\bigskip
Existen dos grandes técnicas para conseguir explotar este paralelismo a nivel de instrucción:

\begin{enumerate}

\item \textbf{Planificación dinámica}

\item \textbf{Planificación estática}

\end{enumerate}

\subsection{Superescalar}

Las máquinas Superescalares hacen uso de la planificación dinámica, es decir, el hardware 
en si es el encargado de mantener la emisión de múltiples instrucciones, decodificarlas y 
ejecutarlas.

\bigskip
Este tipo de parlaleismo conlleva un enorme grado de complejidad, ya que se deben abordar 
todos estos problemas:

\begin{enumerate}

\item Capturar y decodificar más instrucciones en un ciclo.

\item Detectar dependencias entre las instrucciones capturas y las que se ejecutan.

\end{enumerate}

\bigskip
Para resolver los problemas de dependencias se utiliza el algoritmo de Tomasulo.

CITAR ALGORITMO TOMASULO 

\subsection{VLIW}

A diferencia de las máquinas superescalares, las máquinas \textit{Very Long Instruction Word},
mantienen un hardware simple de emisión y todas las responsabilidades caen en el compilador.

\bigskip
Para mantener este hardware simple, estas máquinas trabajan como su propio nombre indica, con
un tamaño de palabra muy grande, es decir, las instrucciones de la máquina son un conjunto
de instrucciones normales que pueden ejecutarse en paralelo.

\bigskip
Esto por supuesto, implica una serie de ventaja / inconvenientes. Por una parte, el compilador,
debe tener una visión completa del programa, conlleva un nivel de complejidad elevado y requiere
un alto grado de especialización por máquina para poder realizar las mayores optimizacions posibles.

\bigskip
Por otra parte, sin embargo, el software es versátil y la circuitería al ser de una menor complejidad 
permite mayores velocidades de reloj.

%---------------------------------------------------------------------------------
\section{Motivación para el trabajo}
\label{1:sec:3}

Como se ha mencionado, el uso de un simulador para apoyar la docencia de esta área
de Arquitecutra de Computadores resultaba un campo realmente interesante. De hecho,
esta herramienta ya existe. El actual profesor de la Universidad de La Laguna 
Iván Castilla desarrolló un simulador en C++ con este propósito.

\bigskip
Este simulador se ha estado utilizando como un complemento más de la docencia, pero
con el paso del tiempo, ha quedado obsoleto. No tanto por su funcionalidad, puesto
que los fundamentos teóricos sobre los que se basa no han cambiado con el tiempo, como 
por su aspecto visual y su accesibilidad.

\bigskip
Es por esto, se ha querido recuperar esta herramienta para continuar con su desarrollo 
y ampliación y este trabajo de fin de grado se centra en migrar esta aplicación a versión 
web de tal forma que sirva como base para los futuros proyectos.

