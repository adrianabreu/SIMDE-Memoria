Debido a la complejidad de las aplicaciones web modernas, es necesario 
realizar una serie de pasos intermedios entre el código original y 
el resultado final de la apliación. Para el caso de este proyecto, se debe:

\begin{itemize}

\item Compilar el código typescript a javascript.

\item Compilar el código .tsx a .jsx.

\item Resolver las importaciones de dependencias, tanto de la lógica
como de los componentes.

\item Procesar el código sass y convertirlo en css.

\end{itemize}

\subsection{Gulp/Grunt}

La primera tendencia -debido a su gran extensión- sería utilizar 
lo que se conoce como un \textit{task runner}. Actualmente, dos de los 
más conocidos son \textbf{Gulp} y \textbf{Grunt}.

\bigskip
Ambos están basados en NodeJs y son compatibles entre sí en gran medida.
Su funcionamiento es sencillo, en un gruntfile o gulpfile se definen las tareas a
ejecutar, seleccionando los ficheros de fuente sobre los que actuar -si cabe- y la tarea 
a realizar.

\bigskip
Existen muchisimos plugins desarrollados que permiten hacer todo tipo de tareas, desde traducir
markdown hasta minimizar el contenido de los ficheros de estilos y de javascript.

\bigskip
Sin embargo, a pesar de que esta opción era altamente atractiva debido 
a su robustez, se ha optado por probar una solución aún más moderna, \textbf{webpack}.

\subsection{Webpack}

Webpack es un module bundler para aplicaciones de Javascript modernas.
Cuando webpack procesa la aplicación, construye un grafo de dependencias
incluyendo todos los módulos.

\bigskip 
El funcionamiento de webpack puede ser toscamente resumido en:

\begin{itemize}

\item Partiendo de un punto de entrada, una serie de reglas activan una serie de loaders
para procesar los distintos tipos de ficheros. 

\item Para el caso de tareas algo más personalizadas y/o complejas, se utilizan plugins especificos.

\end{itemize}

Como resultado final se obtiene una serie de paquetes que contienen todas las dependencias. +

\bigskip 
