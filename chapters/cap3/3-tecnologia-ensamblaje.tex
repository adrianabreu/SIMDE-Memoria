Aunque este proceso pueda parecer trivial, para poder construir el empaquetado final de esta aplicación
deben ejecutarse múltiples tareas:

\begin{enumerate}

\item Compilar el código typescript a javascript.

\item Compilar el código .tsx a .jsx.

\item Resolver los imports de los múltiples componentes.

\item Procesar el código sass y convertirlo en css.

\end{enumerate}

\subsection{Gulp/Grunt}

Con la aparición de NodeJS, se crearon 

\subsection{Webpack}
\bigskip
Una de las grandes herramientas de 2016 que acabó por cambiar el flujo de muchos 
desarrolladores web y desbancó a tasks runners como Grunt y Gulp fue webpack.

\bigskip
Para poder integrar todo este código y resolver el problema de los múltiples imports 
era necesario utilizar algún tipo de herramienta de gestión de paquetes, como por ejemplo 
commonjs, o requirejs. Sin embargo, webpack se encarga de resolver todas estas 
dependencias y crear statics assets para el navegador.

\bigskip
Uno de los mayores puntos a favor es que es altamente configurable, existen muchísimos 
plugins de webpack que permiten hacer preprocesamiento de css, tratamiento de imágenes,
minimización codigo, étctera.. 

\bigskip
Actualmente en la aplicación de SIMDE webpack  se encarga de compilar el código de 
typescript, el código de react y de generar un bundle 