Para integrar la documentación en la nueva aplicación web de SIMDE resultaba obvio que esta documentación
estuviera también en formato web. Para esto existían muchas alternativas, desde un conjunto de ficheros
html hasta un pequeño cms. 

Dado que la documentación es bastante extensa pero que en realidad, no es mas que un documento 
que se redactará en una ocasión y se le irán realizando pequeñas ampliaciones y/o correcciones
se optó por una solución diferente, los generadores de contenido estático.

\subsection{Generadores de contenido estático}

Los generadores de contenido estático se encargan -resumido de forma tosca y breve- de generar 
un conjunto de htmls y css a partir de una plantilla y una serie de ficheros fuentes. 

\bigskip
Este tipo de generadores estáticos tienen un gran auge entre los desarrolladores que desean 
mantneer un blog -yo mismo por ejemplo, tengo uno hecho en Hugo-. 

\bigskip 
Existen múltiples ventajas de utilizar este tipo de tecnologías, pero sin duda para mi la más
importante, es que se alimentan de un formato como es el markdown. El cual es muy intuitivo de 
usar y tiene soporte más allá de este tipo de tecnologías. 

\subsection{Hexo}

Hexo es un generador de contenido estático basado en NodeJS. No posee demasiadas diferencias destacables
sobre el resto y ha sido escogido para este trabajo de fin de grado básicamente por estar basado 
también en Javascript, de tal forma que todo quede enfocado hacia javascript.

\bigskip
¿Comparativa hexo vs jekyll vs hugo?