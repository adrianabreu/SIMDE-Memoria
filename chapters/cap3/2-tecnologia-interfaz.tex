\subsection{Webcomponents}

Desde el inicio del proyecto, tenía claro que alguna librería se encargaría de realizar 
por mí el tedioso proceso de manipular el DOM. Actualmente existen múltiples librerías 
y frameworks que podían servirme en esta tarea, pero muchos de ellos (como por ejemplo Angular),
 no son lo suficientemente flexibles y acaban condicionando el desarrollo de la aplicación.

\bigskip
Lo que SIMDE necesitaba era un proyecto que hiciera uso de los Web components, 
una serie de apis que permiten crear etiquetas html personalizadas y reutilizables. 
Los webs componentes consisten en 4 características principales que pueden usarse 
por separado o todos juntos:

\begin{enumerate}

\item \textbf{Elementos personalizados.}
\item \textbf{Shadow DOM.}
\item \textbf{HTML Imports.}
\item \textbf{HTML Templates.}

\end{enumerate}

\subsection{Polymer}
Probablemente la librería basada en Web Componentes más conocida es Polymer,
una librería desarrollada por Google y anunciada en 2013. 

\bigskip

Pero a pesar de que las múltiples ventajas de Polymer existe una librería cuya comunidad es espléndida,
 y no es otra que React.

\subsection{React}
Existen una gran cantidad de motivos para escoger React sobre Polymer: Como por ejemplo, 
que ahora mismo se utiliza en decenas de aplicaciones importantes, como netflix, airbnb, Wallmart… (TODO INCLUIR CITA)

\bigskip
Que la comunidad es impresionantemente activa y cuenta con una gran cantidad de usuarios dispuestos a ayudar, así como cuenta con muchísima documentación y muchísimas implementaciones de librerías de terceros.

\bigskip
React utiliza un híbrido entre html y javascript denominado jsx, como también tiene soporte para 
Typescript, en este caso utilizamos tsx, y se basa en un “unidirectonial data-flow”. 

\bigskip
Ademas React implementa operaciones sobre el DOM virtual de tal forma que las operaciones sobre
el verdadero DOM sean eficientes.