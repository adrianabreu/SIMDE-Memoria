Desde el inicio del proyecto, se tenía claro que alguna librería se encargaría de realizar 
por mí el tedioso proceso de manipular el DOM. Actualmente existen múltiples librerías 
y frameworks que podían servir para realizar esta tarea, pero muchos de ellos (como por ejemplo Angular),
son demasiado \textit{"rígidos"} y acaban condicionando la forma de desarrollar la aplicación, 
lo cual resulta ser contraproducente.

\subsection{Webcomponents}

\bigskip
Lo que la nueva versión de SIMDE necesitaba era aprovechar las características que ofrecen los
\textit\textbf{Web Components}. Los Web Components son un conjunto de caracteŕisticas que se 
están añadiendo a las especificaciones W3C de Html y del DOM.

\bigskip 
El objetivo de estas características es permitir crear componentes personalizados, reusables y 
con su propia encapsulación. Esto se consigue a través de cuatro características principales:

\begin{enumerate}

\item \textbf{Elementos personalizados}: Esta característica permite diseñar y utilizar nuevos tipos 
de elementos del DOM.
\item \textbf{Shadow DOM}: Esta característica permite al navegador incluir un subarbol de elementos del 
DOM en el renderizado del documento pero \textbf{NO} se incluyen el DOM principal.
\item \textbf{HTML Imports}: Esta característica permite incluir y reutilizar documentos HTML en otros 
documentos HTML.
\item \textbf{Plantillas HTML}: Esta característica permite declarar fragmentos de código de marcas que no
se utilizan en el carga de la página pero que se pueden instanciar en tiempo de ejecución. 

\end{enumerate}

\subsection{Polymer}
La primera librería que apareció haciendo uso de los Web Components fue Polymer. Polymer es desarrollada por
Google y apareció en el año 2013.

\bigskip
Polymer permitía aprovechar las características de los Web Components a traves de polyfills -los polyfills son
códigos que implementan características en navegadores que no soportan las mismas de forma nativa-. Comúnmente
se conoce como polyfill a la librería que implementa el éstandar de HTML5.

\bigskip
Pero hoy día Polymer no es la única opción disponible, y si tuviera que dar una razón para no utilizar Polymer
es que su comunidad no es tan grande como la de otras librerías y eso acaba traduciéndose en una menor cantidad
de recursos.

\subsection{React}
React es una librería desarrollada por Facebook para construir interfaces.

\bigskip
React utiliza un híbrido entre html y javascript denominado jsx, como también tiene soporte para 
Typescript, en este caso utilizamos tsx, y se basa en un “unidirectonial data-flow”. 

\bigskip
Ademas React implementa operaciones sobre el DOM virtual de tal forma que las operaciones sobre
el verdadero DOM sean eficientes.

\bigskip
Existen una gran cantidad de motivos para escoger React sobre Polymer: Como por ejemplo, 
que ahora mismo se utiliza en decenas de aplicaciones importantes, como netflix, airbnb, Wallmart… (TODO INCLUIR CITA)

\bigskip
Que la comunidad es impresionantemente activa y cuenta con una gran cantidad de usuarios dispuestos a ayudar, así como cuenta con muchísima documentación y muchísimas implementaciones de librerías de terceros.
