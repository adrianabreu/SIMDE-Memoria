%%%%%%%%%%%%%%%%%%%%%%%%%%%%%%%%%%%%%%%%%%%%%%%%%%%%%%%%%%%%%%%%%%%%%%%%%%%%%
% Chapter 6: Ampliación de funcionalidades
%%%%%%%%%%%%%%%%%%%%%%%%%%%%%%%%%%%%%%%%%%%%%%%%%%%%%%%%%%%%%%%%%%%%%%%%%%%%%%%

%++++++++++++++++++++++++++++++++++++++++++++++++++++++++++++++++++++++++++++++

En este proyecto de fin de grado se han realizado cuatro ampliaciones sobre las funcionalidades originales 
del simulador. 

%++++++++++++++++++++++++++++++++++++++++++++++++++++++++++++++++++++++++++++++
\section{Documentación}
\label{6:sec1}

La primer ampliación ha sido la incorporación de documentación del programa. Aunque el
término de ampliación no es del todo correcto, puesto que en el proyecto original el autor
elaboró una extensa documentación, esta quedó inaccesible.

\bigskip
La documentación fue realizada en formato .HLP, un formato de ayuda de Windows que quedó
en desuso en Windows Vista. Y esta documentación era realmente interesante, pues no sólo
contenía datos sobre la palicación, sino que incluía consejos para el desarrollo de 
aplicaciones para las distintas máquinas y además explicaba el funcionamiento de las máquinas.

\bigskip
Para recuperar esta documentación se ha utilizado una herramientas de extracción denominada
Help Decompiler. Esta herramienta de línea de comandos procesa los ficheros de ayuda de
Windows .HLP y genera un fichero de texto enriquecido con la documentación y en una carpeta
externa el contenido multimedia que incluye la misma.

\bigskip
Para poder llevar a cabo la tarea de la documentación de forma paralela se consideró que lo mejor era
hacer un proyecto aparte. Resultaba evidente que la documentación de una aplicación web debía de estar en la
web. 

\bigskip
Tras barajar algunas opciones, se optó por mover la documentación a un formato mantenible como 
es markdown. Y partiendo de esto se utilizó un generador de contenido estático basado en NodeJS (Hexo)
para convertir este markdown en web. Se desarrollo un tema simple y personalizado para la ayuda y se 
añadió la capacidad de cambiar entre inglés y español.

\bigskip
Este es el estado actual de la aplicación de la documentación.

    IMAGEN DOCUMENTACIÓN

\bigskip
Se puede acceder a ella desde el menú Ayuda > Documentación en la aplicación del SIMDE.

%++++++++++++++++++++++++++++++++++++++++++++++++++++++++++++++++++++++++++++++
\section{Parseador}
\label{6:sec2}

Una de las características más deseadas por parte de los usuarios de SIMDE 
(entre los que yo mismo me puedo incluir), es un sistema de errores más descriptivo. 

Por desgracia, en la versión original de SIMDE solo se muestra el mensaje "Error".

Ahora, tras una serie de modificaciones en el parseados de código, se muestran los siguientes errores:

\begin{enumerate}
\item Operando erróneo
\item Opcode desconocido
\item Etiqueta repetida
\end{enumerate}

Además, se muestra la línea del error. Esto resulta tremendamente importante, 
ya que una de las aplicaciones de SIMDE requiere realizar mejoras en el rendimiento
de código haciendo uso de técnicas como el desenrollado de bucleas, 
dando lugar a códigos de considerable tamaño.

IMAGENES ERROR


%++++++++++++++++++++++++++++++++++++++++++++++++++++++++++++++++++++++++++++++
\section{Modo de ejecución en lotes}
\label{6:sec3}

Otra característica que habría sido resultado muy deseable por mi parte y por parte de mis compañeros, 
es no tener el limite de velocidad que tiene la versión original.

En mi caso, la cantidad media de ciclos de los ejercicios que se me propusieron superaba los 500 ciclos,
 haciendo un tiempo medio de ejecución de 2 - 3 minutos, lo cual sumado a 
la media de pruebas, a la depuración de errores, y a las distintas optimizaciones, 
alargaba de forma innecesaria el desarrollo de los ejercicios.

Es por eso que ahora, cuando el campo de velocidad está en 0, se ejecuta la simulación a velocidad 
máxima, y solo se refresca la interfaz cuando termina la ejecución. Ahorrando así 
recursos. 

%++++++++++++++++++++++++++++++++++++++++++++++++++++++++++++++++++++++++++++++
\section{Histórico}
\label{6:sec4}

Por costumbre, en las primeras clases de SIMDE se utilizaba de forma 