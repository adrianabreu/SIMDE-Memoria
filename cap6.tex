%%%%%%%%%%%%%%%%%%%%%%%%%%%%%%%%%%%%%%%%%%%%%%%%%%%%%%%%%%%%%%%%%%%%%%%%%%%%%
% Chapter 6: Ampliación de funcionalidades
%%%%%%%%%%%%%%%%%%%%%%%%%%%%%%%%%%%%%%%%%%%%%%%%%%%%%%%%%%%%%%%%%%%%%%%%%%%%%%%

%++++++++++++++++++++++++++++++++++++++++++++++++++++++++++++++++++++++++++++++

En este proyecto de fin de grado se han realizado cuatro ampliaciones de las funcionales. 

%++++++++++++++++++++++++++++++++++++++++++++++++++++++++++++++++++++++++++++++
\section{Documentación}
\label{6:sec1}

El primero de ellos es recuperación de la documentación. En su proyecto original el profesor Iván Castilla elaboró una extensa documentación sobre SIMDE que contenía consejos sobre el desarrollo de aplicaciones para las distintas máquinas y no solo eso, sino que además explicaba el funcionamiento de las máquinas.

Por desgracia, esta documentación fue realizada en formato .HLP, un formato de windows que quedó en desuso desde Windows Vista. Es por eso, que utilizando herramientas de extracción, se pudieron recuperar los textos originales de parte del profesor.

Para integrar integrar esta documentación en la aplicación web lo mejor era hacer un sistema web también. Por ello se ha migrado la documentación a un formato muchísimo más mantenible como es markdown y utilizando un generador de contenido estático basado en NodeJS (Hexo) y un tema apropiado, se ha conseguido que la documentación esté integrada de nuevo.

    IMAGEN DOCUMENTACIÓN

%++++++++++++++++++++++++++++++++++++++++++++++++++++++++++++++++++++++++++++++
\section{Parseador}
\label{6:sec2}

Una de las características más deseadas por parte de los usuarios de SIMDE (entre los que yo mismo me puedo incluir), es un sistema de errores más descriptivo. 

Por desgracia, en la versión original de SIMDE solo se muestra el mensaje "Error".

Ahora, tras una serie de modificaciones en el parseados de código, se muestran los siguientes errores:

\begin{enumerate}
\item Operando erróneo
\item Opcode desconocido
\item Etiqueta repetida
\end{enumerate}

Además, se muestra la línea del error. Esto resulta tremendamente importante, ya que una de las aplicaciones de SIMDE requiere realizar mejoras en el rendimiento
de código haciendo uso de técnicas como el desenrollado de bucleas, dando lugar a códigos de considerable tamaño.

IMAGENES ERROR


%++++++++++++++++++++++++++++++++++++++++++++++++++++++++++++++++++++++++++++++
\section{Batch mode}
\label{6:sec3}

Otra característica que habría sido resultado muy deseable por mi parte y por parte de mis compañeros, es no tener el limite de velocidad que tiene la versión original.

En mi caso, la cantidad media de ciclos de los ejercicios que se me propusieron superaba los 500 ciclos, haciendo un tiempo medio de ejecución de 2 - 3 minutos, lo cual sumado a 
la media de pruebas, a la depuración de errores, y a las distintas optimizaciones, alargaba de forma innecesaria el desarrollo de los ejercicios.

Es por eso que ahora, cuando el campo de velocidad está en 0, se ejecuta la simulación a velocidad máxima, y solo se refresca la interfaz cuando termina la ejecución. Ahorrando así 
recursos. 

%++++++++++++++++++++++++++++++++++++++++++++++++++++++++++++++++++++++++++++++
\section{Histórico}
\label{6:sec4}

